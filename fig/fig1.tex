\documentclass[tikz]{standalone}

\usepackage[english,russian]{babel}
\usepackage[T2A,T1]{fontenc}
\usepackage[utf8x]{inputenc}
\usepackage
    {
        tikz,
        pgfplots,
        verbatim,
        tikz-3dplot 
    }
\usetikzlibrary
    {
        arrows,
        patterns,
        angles,
        quotes,
        calc, 
        3d, 
        backgrounds, 
        positioning,
        babel
    }
\usepackage{ifthen}
\newcommand{\lineann}[4][0.5]{%
    \begin{scope}[rotate=#2, blue,inner sep=2pt, ]
        \draw[dashed, blue!40] (0,0) -- +(0,#1)
            node [coordinate, near end] (a) {};
        \draw[dashed, blue!40] (#3,0) -- +(0,#1)
            node [coordinate, near end] (b) {};
        \draw[|<->|] (a) -- node[fill=white, scale=0.8] {#4} (b);
    \end{scope}
}
\newcommand{\polarize}[4][magenta]{%
	\begin{scope}[xshift={#2 cm}, yshift=0.75cm]
		\draw[dashed,->] (0,0.75) -- ++(90:0.75) node[above] {$y$};
		\draw[dashed,->] (0,0.75) -- ++(0:0.75) node[right] {$x$};
		\draw[dashed] (0,0.75) -- ++(#3:0.75);
		\draw (0,0.75) circle (0.3cm);
        \xdef\angle{#3}
        \draw [-,#1, line width=1pt] (0,0.75) -- ++({180+\angle}:0.3);
        \draw [->,#1, line width=1pt] (0,0.75) -- ++({\angle}:0.3);

        \ifthenelse{\angle=0 \OR \angle=90}{}
        {
			\draw[dashed] (0,0.75) coordinate (a) node[right] {}
			-- (0,0.75) coordinate (b) node[left] {}
			-- (0,0.75)++(#3:1) coordinate (c) node[above right] {}
			pic["$#4$", draw=black, <-, solid, angle eccentricity=1.5, angle radius=0.7cm]
				{angle=c--b--a};
        }

	\end{scope}
}


\newcommand{\polvec}[2]{%
	\begin{scope}[xshift={#1 cm}, yshift=0cm]
		\xdef\angle{#2}
		\draw [-,magenta, line width=1pt] (0,0.75) -- ++({180+90}:{0.25*cos(\angle)});
		\draw [->,magenta, line width=1pt] (0,0.75) -- ++({90}:{0.25*cos(\angle)});
	\end{scope}
}


\begin{document}

\begin{tikzpicture}[scale=1]

    \xdef\setka{0}
	\xdef\opa{0.1}
	\xdef\SIZE{8} 
	\xdef\SIZEX{4}
	\ifx\setka\undefined
    {}
    \else
      \if\setka1
        \draw[step=1.0,blue,thick, opacity=\opa] (0,0) grid (\SIZE,\SIZEX);
        \draw[step=0.5,blue,very thin, opacity=\opa] (0,0) grid (\SIZE,\SIZEX);

        \foreach \i in {0,1,...,\SIZE} {
            \draw (\i,0) node [below,  opacity=\opa] {\i};
        }
        \foreach \i in {0,1,...,\SIZEX} {
            \draw (0,\i) node [left,  opacity=\opa] {\i};
            % \draw (0,\i) node [left,  opacity=\opa] {\i};
        }        
      \else
        {}
      \fi
    \fi

	\draw[thick,->] (0,0) -- node[above,sloped] {Энергия электрона} (0,4);

	\draw[thick,->] (1,0.5) -- node[below,sloped] {Металл} (3,0.5);
	\draw[thick,->] (5,0.5) -- node[below,sloped] {Вакуум} (3,0.5);


	\draw[thick,<->] (4,2) -- node[right] {$e\varphi$} (4,3);
	\draw[thick,<->] (4,3) -- node[right] {$W_{max}$} (4,4);

	\draw[thick,<->] (2,2) -- node[left] {$h\nu$} (2,4);
	% \draw (3,-0) -- (3.5,2) -- (3,4) --cycle;

	\draw[pattern=north east lines, pattern color=blue, draw=none] (1,1) -- (1,2) -- (3,2) -- (3,1) --cycle;

	\draw[thick] (1,2) -- (3,2);
	\draw[thick] (1,1) -- (3,1);

	\draw[thick] (3,0) -- (3,3) -- (5,3);
	\draw[thick, dashed] (3,2) -- (5,2);

	\draw[thick, dashed] (1,4) -- (5,4);

	\node[inner sep=0pt] (russell) at (0,0)
    {\includegraphics[width=.015\textwidth]{fg.png}};


	% \draw[fill=magenta] (0,2) circle (2pt) node [above] {$S$};
	% \draw[fill=magenta] (0,3.5)  circle (2pt) node [above] {$S_1$};
	% \draw[fill=magenta] (0,0.5)  circle (2pt) node [above] {$S_2$};

	% \draw [dashdotted, opacity=0.4] (0,2) -- ++ (8,0);

	% \draw [dashed] (0,3.5) -- (3,4) coordinate (k) -- ++({atan(1/6)}:4.09) coordinate (A);
	% \draw [dashed] (0,0.5) -- (3,0)  coordinate (k')-- ++({atan(-1/6)}:4.09) coordinate (A');

	% \draw (0,2)  coordinate (oo) -- (3,4) coordinate (1); \draw [dashed] (3,4) -- ++({atan(2/3)}:1) coordinate (K);
	% \draw (0,2) -- (3,0) coordinate (2); \draw [dashed] (3,0) -- ++({atan(-2/3)}:1) coordinate (K');

	% \draw [dashed] (0,3.5) -- (3.5,2) coordinate (o);
	% \draw [dashed] (0,0.5) -- (3.5,2);
	% \draw (o) -- ++({atan(3/7)}:3.8) coordinate (B);
	% \draw (o) -- ++({atan(-3/7)}:3.8) coordinate (B');

	% \draw[draw=none, fill=blue, opacity=0.3] (o) -- (3,4) -- (A) -- (B') -- cycle;
	% \draw[draw=none, fill=red, opacity=0.3] (o) -- (3,0) -- (A') -- (B) -- cycle;

	% \draw pic["$\varphi_0$", draw=black, <->, solid, angle eccentricity=1.5, angle radius=0.7cm]
	% 			{angle=2--oo--1};
	% \draw pic["$2\varepsilon$", draw=black, <->, solid, angle eccentricity=1.2, angle radius=1.5cm]
	% 			{angle=B'--o--B};

	% \draw pic["$\varepsilon$", draw=black, <->, solid, angle eccentricity=1.5, angle radius=0.7cm]
	% 			{angle=A--k--K};


	% \draw pic["$\alpha$", draw=black, <->, solid, angle eccentricity=1.2, angle radius=1.3cm]
	% 			{angle=k'--k--o};				
	% % \draw [fill=white, draw=none] (7,-1) rectangle (7.5,5);
	% \draw [thick, ->] (7,-1) -- (7,5) node [right] {$x$};

	% \begin{scope}[yshift=2cm]
	% 	\lineann[-3.5]{0}{3.5}{$h$}
	% \end{scope}
	% \begin{scope}[yshift=0cm]
	% 	\lineann[-1.4]{0}{7}{$L$}
	% \end{scope}	

	% \draw (7,3.5) node [right] {$M$};
	% \draw (7,0.5) node [right] {$N$};
	% \draw (0,2) node [yshift=-.32em, scale=1.3] {*} node [above] {$S$};
	% \begin{scope}[xshift=-0.5cm]
	% 	\draw (0,0.5) rectangle node[above, yshift=1em] {1} ++(0.5,0.5);
	% \end{scope}

	% \draw[yshift=0.75cm,->, dashed] (-1.5,0) -- ++(0.6,0) node[right] {$z$};
	% \draw[yshift=0.75cm,->, dashed] (-1.5,0) -- ++(0,0.6) node[above] {$y$};
	% \draw[yshift=0.75cm,->, dashed] (-1.5,0) node [solid, fill=white] {$\bigotimes$} node[below, yshift=-0.5em] {$x$};

	% \begin{scope}[xshift=0.5cm]
	% 	\draw (0,0.5) rectangle node[above, yshift=1em] {2} ++(0.5,0.5);
	% 	\draw (0,0.5) -- ++ (0.5,0.5);
	% \end{scope}
	% \draw[magenta, line width=1pt] (0,0.75) -- ++ (0.5,0);

	% \draw[magenta, line width=1pt] (1,0.75) -- ++ (1,0);
	% \draw[magenta, line width=1pt] (2,0.75) -- ++ (1,0);
	% \draw[magenta, line width=1pt] (4,0.75) -- ++ (2.5,0);

	% \xdef\angle{90}
	% \draw [-,magenta, line width=1pt] (2,0.75) -- ++({180+\angle}:0.25);
	% \draw [->,magenta, line width=1pt] (2,0.75) -- ++({\angle}:0.25);
	% \draw[draw=none] (0,0) -- ++(6.75,0);
	% % \draw[dashed] (2,0.75) -- ++(90:1);

	% \polarize[magenta]{2}{90}{\frac{\pi}{2}}
	% \polarize[magenta]{5}{45}{\Theta}

	% \polvec{5}{45}

	% \begin{scope}[xshift=3cm, yshift=0.75cm]
	% 	\draw[line width=3pt] (0,-0.25) -- ++ (1,0);
	% 	\draw[line width=3pt] (0,0.25) -- node[above, yshift=0.2em] {3}  ++ (1,0);
	% 	\draw[magenta, line width=1pt] (0,0) -- (1,0);
	% 	\draw[fill=white] (0.1,-0.1) rectangle ++ (0.8,0.2);
	% \end{scope}

\end{tikzpicture}
\end{document}